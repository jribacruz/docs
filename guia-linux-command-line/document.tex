%%This is a very basic article template.
%%There is just one section and two subsections.
\documentclass[a4paper,9pt]{extarticle}
\usepackage[utf8]{inputenc}
\renewcommand*\rmdefault{ppl} 
\usepackage{geometry}
\geometry{margin=.4in}

\usepackage{multicol}
\usepackage{pdflscape}


%%Cria os índices
\usepackage{hyperref}
\hypersetup{colorlinks=true,allcolors=blue}
\usepackage{hypcap} 

\usepackage{listings} 

%%Configuração das seções
\usepackage[dvipsnames]{xcolor}
\usepackage{titlesec}
\titleformat{\section}
{\color{Red}\normalfont\Large\bfseries}
{\color{Red}\thesection}{1em}{}



\begin{document}
\title{Guia Linux / Conceitos e Comandos}
\author{José Ribamar Monteiro da Cruz}
\maketitle
\tableofcontents



%\begin{landscape}
\begin{multicols}{2}
%%
\section{Gerenciamento de Pacotes}
\subsection{Configurando proxy}
	Abrir o arquivo para edição. Criar caso não exista.
	
	\paragraph{\fbox{\texttt{\$ sudo nano /etc/dnf/dnf.conf}}} \emph{Fedora}
	
	\paragraph{\fbox{\texttt{\$ sudo nano /etc/apt/apt.conf}}} \emph{Debian/Ubuntu} \newline
	
	\noindent Inserir as seguintes informações e salvar arquivo.
	
	\paragraph{\fbox{\texttt{Acquire::http::Proxy "http://\emph{USUARIO}:\emph{SENHA}@\emph{IP}:\emph{PORTA}";}}} \emph{Debian/Ubuntu}
	
	
\subsection{Atualizando pacotes}
	
	\paragraph{\fbox{\texttt{\$ sudo dnf update}}} \emph{Fedora}
	
	\paragraph{\fbox{\texttt{\$ sudo apt-get upgrade}}} \emph{Debian/Ubuntu}
	
\subsection{Buscando pacotes}
	
	\paragraph{\fbox{\texttt{\$ dnf search \emph{NOME-DO-PACOTE}}}} \emph{Fedora}
	
	\paragraph{\fbox{\texttt{\$ apt-cache search \emph{NOME-DO-PACOTE}}}} \emph{Debian/Ubuntu}
	
\subsection{Instalando pacotes}

	\paragraph{\fbox{\texttt{\$ sudo dnf install \emph{NOME-DO-PACOTE}}}} \emph{Fedora}
	
	\paragraph{\fbox{\texttt{\$ sudo apt-get install \emph{NOME-DO-PACOTE}}}} \emph{Debian/Ubuntu}
	
\subsection{Listando pacotes instalados}
	
	\paragraph{\fbox{\texttt{\$ dnf list}}} \emph{Fedora}
	
\subsection{Exibindo informações de pacote}
	
	\paragraph{\fbox{\texttt{\$ dnf info \emph{NOME-DO-PACOTE}}}} \emph{Fedora}
	
	\paragraph{\fbox{\texttt{\$ apt-cache show \emph{NOME-DO-PACOTE}}}} \emph{Debian/Ubuntu}
	
\subsection{Removendo pacote}
	\paragraph{\fbox{\texttt{\$ sudo dnf remove \emph{NOME-DO-PACOTE}}}} \emph{Fedora}
	
	\paragraph{\fbox{\texttt{\$ sudo apt-get remove \emph{NOME-DO-PACOTE}}}} \emph{Debian/Ubuntu}
%%
\section{Arquivos e Diretórios}
\subsection{Exibindo diretório atual}

	\fbox{\texttt{\$ pwd}}
	
\subsection{Listando arquivos de um diretório}
	
	\paragraph{\fbox{\texttt{\$ ls -l}}} \emph{1ª coluna são os conjuntos de permissões.}
	\paragraph{\fbox{\texttt{\$ ls -la}}} \emph{Inclui arquivos ocultos.}
	\paragraph{\fbox{\texttt{\$ ls -l abc*}}} \emph{Arquivos que iniciam com abc.}
	\paragraph{\fbox{\texttt{\$ ls -l *.log}}} \emph{Arquivos com extensão log.}
	
	
\subsection{Navegando entre diretórios}

	\paragraph{\fbox{\texttt{\$ cd \emph{/DIR-DEST} }}}
	\paragraph{\fbox{\texttt{\$ cd \~}}} \emph{Vai para home do usuário logado.}
	\paragraph{\fbox{\texttt{\$ cd ..}}} \emph{Retorna para diretório pai.}
	
	
\subsection{Criando diretório}

	\fbox{\texttt{\$ mkdir \emph{NOME-DIR}}}

\subsection{Removendo diretório}

	\fbox{\texttt{\$ rmdir \emph{NOME-DIR}}}
	
\subsection{Criando arquivo}
	
	\fbox{\texttt{\$ touch \emph{/DIR/ARQ.xxx}}}
	
\subsection{Removendo arquivo}
	
	\paragraph{\fbox{\texttt{\$ rm /DIR/ARQ.xxx}}}
	
\subsection{Cópia de arquivos e diretórios} 
\subsubsection{Mesmo host}
	
	\paragraph{\fbox{\texttt{\$ cp \emph{/DIR-ORIG/ARQ.xxx /DIR-DEST}}}}
	
	
	\paragraph{\fbox{\texttt{\$ cp -R \emph{/DIR-ORIG/ /DIR-DEST}}}} \emph{Todo conteúdo do diretório.} 
	
\subsubsection{Entre hosts diferentes}
	
	\fbox{\texttt{\$ scp \emph{/DIR-ORIG/ARQ.xxx USER@IP:/DIR-DEST}}}

\subsection{Movendo arquivo} 

   \fbox{\texttt{\$ mv \emph{/DIR-ORIG/ARQ.xxx /DIR-DEST}}}

\subsection{Renomeando arquivo}

 	\fbox{\texttt{\$ mv \emph{/DIR-ORIG/ARQ.xxx /DIR-DEST/NOVO-ARQ.xxx}}}
 	

\subsection{Exibindo espaço utilizado em diretório}
	
	\paragraph{\fbox{\texttt{\$ du -sh }}} \emph{Formato sumarizado.}
	\paragraph{\fbox{\texttt{\$ du -ah }}} \emph{Formato por arquivo e diretório.}
	
\subsection{Alterando proprietário de arquivo}

	\fbox{\texttt{\$ chown \emph{USER}:\emph{GROUP} \emph{ARQ.xxx}}}
	
\subsection{Alterando permissões de arquivo}

	\paragraph{\fbox{\texttt{\$ chmod 755 \emph{/DIR/ARQ.xxx}}}} \emph{Todos os privilégios ao usuário.}
	
	\paragraph{\fbox{\texttt{\$ chmod +x \emph{/DIR/ARQ.xxx}}}} \emph{Permissão de execução.}

\subsection{Editando arquivo}
	
	\paragraph{\fbox{\texttt{\$ vi \emph{/DIR/ARQ.xxx}}}} \emph{Editor vi. Tecla i para modo de edição.}
	
	\paragraph{\fbox{\texttt{\$ nano \emph{/DIR/ARQ.xxx}}}} \emph{Editor nano.}

\subsection{Salvando edição em arquivo} 
	
	\paragraph{\fbox{\texttt{\emph{ESC} + :wq}}} \emph{Editor vi. Salva e sai do editor.}
	
	\paragraph{\fbox{\texttt{\emph{ESC} + :q}}} \emph{Editor vi. Sai sem salvar.}
	
	\paragraph{\fbox{\texttt{\emph{CTRL+X} + s + ENTER}}} \emph{Editor nano. Salva e sai do editor.}
	
	\paragraph{\fbox{\texttt{\emph{CTRL+X} + n}}} \emph{Editor nano. Sai sem salvar.}
	
\subsection{Busca de arquivo em diretório}
	
	\paragraph{\fbox{\texttt{\$ find . -name \emph{ARQ.xxx}}}} \emph{Busca por ARQ.xxx no diretório atual.}
	
	\paragraph{\fbox{\texttt{\$ find /home -name \emph{ARQ.xxx}}}} \emph{Busca por ARQ.xxx em /home.}
	
\subsection{Verificando o charset e mime de arquivo}
	
	\fbox{\texttt{\$ file -bi \emph{/DIR/ARQ.xxx}}}
		
\subsection{Calculando o hash (MD5) de um arquivo}

	\paragraph{\fbox{\texttt{\$ md5sum \emph{/DIR/ARQ.xxx}}}}
	
\subsection{Exibindo conteúdo de arquivo}
\subsection{Dividindo arquivo}


\subsection{Contando número de linhas de arquivo}
	
	\paragraph{\fbox{\texttt{\$ wc \emph{/DIR/ARQ.xxx}}}} \emph{Nº de linhas na 1ª coluna.}
	
\subsection{Compactando diretórios (zip)}
	
	\fbox{\texttt{\$ zip -r \emph{ARQUIVO.zip /DIR-A-SER-ZIPADO}}}

\subsection{Descompactando arquivos zipados}
	
	\fbox{\texttt{\$ unzip \emph{ARQUIVO.zip}}} \emph{Descompacta no diretório atual.}
l
%%
\section{Processos}
\subsection{Listando processos}
	
	\paragraph{\fbox{\texttt{\$ ps -aux }}} \emph{A segunda coluna é o PID do processo.} 
	
\subsection{Matando processo}
	\fbox{\texttt{\$ kill -9 \emph{PID-DO-PROCESSO}}}
%
\section{Rede}
\subsection{Verificando o IP e MAC da máquina}

	\fbox{\texttt{\$ sudo ifconfig}}
	
\subsection{Verificando os serviços da máquina}
	
	\fbox{\texttt{\$ netstat -ant}}

\subsection{Realizando download de conteúdo web}
	 
	\fbox{\texttt{\$ wget \emph{http://ENDERECO-WEB/conteudo-a-ser-baixado.xxx}}}
	
\subsection{Realizando requisição HTTP}
\subsection{Analizando trafego de rede por aplicação}

%%
\section{Sistema}
\subsection{Exibindo uso de RAM}
	
	\fbox{\texttt{\$ free -h}}
	
\subsection{Exibindo partições e espaço em disco}
	
	\fbox{\texttt{\$ df -h}}

\subsection{Exibindo a data do sistema}
	
	\fbox{\texttt{\$ date}}
	
\subsection{Alterando a data do sistema}
	
	\fbox{\texttt{\$ sudo date -s "yyyy-MM-dd HH:mm:ss"}}
	
\subsection{Exibindo versão do kernel e do sistema}
	
	\fbox{\texttt{\$ uname -a}}
	
\subsection{Reiniciando a máquina}
	
	\fbox{\texttt{\$ sudo reboot}}
	
\subsection{Listando dispositivos PCI}
	
	\fbox{\texttt{\$ lspci}}
	
\subsection{Exibindo usuários logado na máquina}

	\fbox{\texttt{\$ w}}
	
\subsection{Exibindo histórico de comandos executados} 
	
	\paragraph{\fbox{\texttt{\$ history}}} \emph{A primeiro coluna é o número do histórico.}
	
\subsection{Executando comando do histórico}

	\fbox{\texttt{\$ !\emph{NUMERO-DO-HISTORICO}}}


\end{multicols}
%\end{landscape}
\end{document}
