%%This is a very basic article template.
%%There is just one section and two subsections.
\documentclass[a4paper,8pt]{extarticle}
\usepackage[utf8]{inputenc}
\renewcommand*\rmdefault{ppl}
\usepackage{geometry}
\geometry{margin=.5in}

\usepackage{multicol}


%%Cria os índices
\usepackage{hyperref}
\hypersetup{colorlinks=true,allcolors=blue}
\usepackage{hypcap} 

\usepackage{listings}
\usepackage{courier}

\begin{document}
\title{Guia Linux - Linha de Comando}
\author{José Ribamar Monteiro da Cruz}
\maketitle
\tableofcontents



\begin{multicols}{2}
%%
\section{Gerenciamento de Pacotes}
\subsection{Configurando proxy}
\subsection{Atualizando lista de pacotes}
\subsection{Buscando pacotes}
\subsection{Instalando pacotes}
\subsection{Listando pacotes instalados}
\subsection{Exibindo informações de pacote}
\subsection{Removendo pacote}
 
%%
\section{Arquivos e Diretórios}
\subsection{Exibindo diretório atual}
\subsection{Listando arquivos de um diretório}
\subsection{Navegando entre diretórios}
\subsection{Criando diretório}
\subsection{Removendo diretório}
\subsection{Criando arquivo}
\subsection{Removendo arquivo}
\subsection{Copiando arquivos}
\subsubsection{Mesmo host}
\subsubsection{Entre hosts diferentes}
\subsection{Movendo arquivos} 
	

   \texttt{\$ mv /DIR/ORIGEM/ARQUIVO /DIR/DESTINO}

\subsection{Renomeando arquivo}

\begin{lstlisting}[language=bash]
  $ mv /DIR/ORIGEM/ARQUIVO /DIR/DESTINO/NOVO_ARQUIVO
\end{lstlisting}

\subsection{Exibindo espaço utilizado em diretório}
\subsection{Alterando proprietário de arquivo}
\subsection{Alterando permissões de arquivo}
\subsection{Busca de arquivo em diretório}
\subsection{Verificando o charset de arquivo}
\subsection{Calculando o hash (MD5) de um arquivo}
Uma forma simples de verificar ou garantir a integridade de um arquivo é
calculando o seu respectivo hash(MD5).
\subsection{Exibindo conteúdo de arquivo}
\subsection{Dividindo arquivo}
\subsection{Editando arquivo}
\subsection{Compactando diretórios}


%%
\section{Processos}
\subsection{Listando processos}
\subsection{Matando processos}

%
\section{Rede}
\subsection{Verificando o IP da máquina}
\subsection{Verificando os serviços da máquina}
\subsection{Realizando download de conteúdo web}
\subsection{Realizando requisição HTTP}
\subsection{Analizando trafego de rede por aplicação}

%%
\section{Sistema}
\subsection{Exibindo uso de RAM}
\subsection{Exibindo partições e espaço em disco}
\subsection{Exibindo a data do sistema}
\subsection{Alterando a data do sistema}
\subsection{Exibindo versão do kernel e do sistema}
\subsection{Reiniciando a máquina}
\subsection{Listando dispositivos PCI}
\subsection{Exibindo usuários logado na máquina}
\subsection{Habilitando sudo}

\section{Diversos}
\subsection{Exibindo histórico de comandos}
\subsection{Executando comando do histórico}



\end{multicols}
\end{document}
