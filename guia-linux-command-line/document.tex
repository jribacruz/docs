%%This is a very basic article template.
%%There is just one section and two subsections.
\documentclass[a4paper,9pt]{extarticle}
\usepackage[utf8]{inputenc}
\renewcommand*\rmdefault{ppl} 
\usepackage{geometry}
\geometry{margin=.5in}

\usepackage{multicol}
\usepackage{pdflscape}


%%Cria os índices
\usepackage{hyperref}
\hypersetup{colorlinks=true,allcolors=blue}
\usepackage{hypcap} 

\usepackage{listings}

%%Configuração das seções
\usepackage[dvipsnames]{xcolor}
\usepackage{titlesec}
\titleformat{\section}
{\color{Red}\normalfont\Large\bfseries}
{\color{Red}\thesection}{1em}{}



\begin{document}
\title{Guia Linux - Linha de Comando}
\author{José Ribamar Monteiro da Cruz}
\maketitle
\tableofcontents



%\begin{landscape}
\begin{multicols}{2}
%%
\section{Gerenciamento de Pacotes}
\subsection{Configurando proxy}
\subsection{Atualizando lista de pacotes}
\subsection{Buscando pacotes}
\subsection{Instalando pacotes}
\subsection{Listando pacotes instalados}
\subsection{Exibindo informações de pacote}
\subsection{Removendo pacote}
 
%%
\section{Arquivos e Diretórios}
\subsection{Exibindo diretório atual}

	\fbox{\texttt{\$ pwd}}
	
\subsection{Listando arquivos de um diretório}
	
	\paragraph{\fbox{\texttt{\$ ls -l}}}
	\paragraph{\fbox{\texttt{\$ ls -la}}} \emph{Inclui arquivos ocultos.}
	
	
\subsection{Navegando entre diretórios}

	\fbox{\texttt{\$ cd \emph{/DIR-DEST} }}
	
\subsection{Criando diretório}

	\fbox{\texttt{\$ mkdir \emph{NOME-DIR}}}

\subsection{Removendo diretório}
\subsection{Criando arquivo}
	
	\fbox{\texttt{\$ touch \emph{ARQ.xxx}}}
	
\subsection{Removendo arquivo}
\subsection{Copiando arquivo} 
\subsubsection{Mesmo host}
	
	\fbox{\texttt{\$ cp \emph{/DIR-ORIGEM/ARQ.xxx /DIR-DEST}}}
	
\subsubsection{Entre hosts diferentes}
	
	\fbox{\texttt{\$ scp \emph{/DIR-ORIGEM/ARQ.xxx USER@IP:/DIR-DEST}}}

\subsection{Movendo arquivo} 
	

   \fbox{\texttt{\$ mv \emph{/DIR-ORIGEM/ARQ.xxx /DIR-DEST}}}

\subsection{Renomeando arquivo}

 	\fbox{\texttt{\$ mv \emph{/DIR-ORIGEM/ARQ.xxx /DIR-DEST/NOVO-ARQ.xxx}}}
 	

\subsection{Exibindo espaço utilizado em diretório}
	
	\paragraph{\fbox{\texttt{\$ du -sh }}} \emph{Formato sumarizado.}
	\paragraph{\fbox{\texttt{\$ du -ah }}} \emph{Formato por arquivo e diretório.}
	
\subsection{Alterando proprietário de arquivo}

	\fbox{\texttt{\$ chown \emph{USER}:\emph{GROUP} \emph{ARQ.xxx}}}
	
\subsection{Alterando permissões de arquivo}

	\paragraph{\fbox{\texttt{\$ chmod 755 \emph{ARQ.xxx}}}} \emph{Todos os privilégios ao usuário.}
	
\subsection{Busca de arquivo em diretório}
\subsection{Verificando o charset e mime de arquivo}
	
	\fbox{\texttt{\$ file -bi}}
		
\subsection{Calculando o hash (MD5) de um arquivo}

	\paragraph{\fbox{\texttt{\$ md5sum \emph{ARQ.xxx}}}}
	
\subsection{Exibindo conteúdo de arquivo}
\subsection{Dividindo arquivo}
\subsection{Editando arquivo}
\subsection{Compactando diretórios}
	
	\fbox{\texttt{\$ zip -r \emph{ARQUIVO.zip /DIR-A-SER-ZIPADO}}}
	

%%
\section{Processos}
\subsection{Listando processos}
	
	\paragraph{\fbox{\texttt{\$ ps -aux }}} \emph{A segunda coluna é o PID do processo.} 
	
\subsection{Matando processo}
	\fbox{\texttt{\$ kill -9 \emph{PID-DO-PROCESSO}}}
%
\section{Rede}
\subsection{Verificando o IP e MAC da máquina}

	\fbox{\texttt{\$ sudo ifconfig}}
	
\subsection{Verificando os serviços da máquina}
	
	\fbox{\texttt{\$ netstat -ant}}

\subsection{Realizando download de conteúdo web}
	 
	\fbox{\texttt{\$ wget \emph{http://context:port/conteudo-a-ser-baixado.xxx}}}
	
\subsection{Realizando requisição HTTP}
\subsection{Analizando trafego de rede por aplicação}

%%
\section{Sistema}
\subsection{Exibindo uso de RAM}
	
	\fbox{\texttt{\$ free -h}}
	
\subsection{Exibindo partições e espaço em disco}
	
	\fbox{\texttt{\$ df -h}}

\subsection{Exibindo a data do sistema}
	
	\fbox{\texttt{\$ date}}
	
\subsection{Alterando a data do sistema}
\subsection{Exibindo versão do kernel e do sistema}
	
	\fbox{\texttt{\$ uname -a}}
	
\subsection{Reiniciando a máquina}
	
	\fbox{\texttt{\$ sudo reboot}}
	
\subsection{Listando dispositivos PCI}
	
	\fbox{\texttt{\$ lspci}}
	
\subsection{Exibindo usuários logado na máquina}

	\fbox{\texttt{\$ w}}
	
\subsection{Exibindo histórico de comandos executados} 
	
	\paragraph{\fbox{\texttt{\$ history}}} \emph{A primeiro coluna é o número do histórico.}
	
\subsection{Executando comando do histórico}

	\fbox{\texttt{\$ !\emph{NUMERO-DO-HISTORICO}}}


\end{multicols}
%\end{landscape}
\end{document}
